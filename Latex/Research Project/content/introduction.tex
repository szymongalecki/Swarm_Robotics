\section{Introduction}
In swarm robotics, multiple robots work together to solve problems by interacting with each other and the environment in a similar way as bees, ants or birds. \cite{Swarm_Robotic_Behaviors_and_Current_Applications}. To enable the interaction within the swarm, first we have to achieve aggregation. There are many algorithms that focus on aggregation task for robot swarm but we have chosen the Alpha algorithm as it was mentioned in reviewed papers \cite{Towards_Temporal_Verification_of_Emergent_Behaviours_in_Swarm_Robotic_Systems}, \cite{On_Formal_Specification_of_Emergent_Behaviours_in_Swarm_Robotic_Systems}, \cite{Symmetry_Reduction_Enables_Model_Checking_of_More_Complex_Emergent_Behaviours_of_Swarm_Navigation_Algorithms}, \cite{A_Matrix_Based_Approach_For_Modeling_Robotic_Swarm_Behavior}, \cite{Verification_of_visibility-based_properties_on_multiple_moving_robots_in_an_environment_with_obstacles} and because it achieves aggregation using a single variable.

Instead of relying on environment and localisation information, it uses physical properties of the signal used for communication. Robot behaviour is solely determined by the change in the number of robots that are in the range of its signal.

The idea behind the Alpha algorithm is that aggregation can be achieved using just the information on the number of robot connections and completely disregarding the environment in which robots exist. The physical reality of such solution would employ a communication technology of a limited range to establish connections to other robots. A single robot behaviour would be determined solely on the number of connections.

In reviewed papers the modeling and implementation parts were either limited or missing. That means that any future work or repeating experiments based on those papers is not possible. This paper will focus on the modeling and implementation aspect of the Alpha algorithm. We will show how an algorithm defined by a pseudocode is transformed into a timed automaton. Using timed automaton will allow for expressing time-dependent behaviours and verifying the model that better captures the real world operation. We will use an integrated tool for modeling and verification, UPPAAL, to implement a timed automaton. We will create system of timed automata that implements the Alpha algorithm for robot swarm. Finally, we will examine the correctness of our implementation by defining and verifying properties of the system.