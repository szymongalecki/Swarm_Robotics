\section{Conclusions}
In this paper we have presented the detailed implementation of the Alpha algorithm \cite{Minimalist_Coherent_Swarming_of_Wireless_Networked_Autonomous_Mobile_Robots} using network of timed automata. The Alpha algorithm has been chosen because of its presence in reviewed literature \cite{Towards_Temporal_Verification_of_Emergent_Behaviours_in_Swarm_Robotic_Systems}, \cite{On_Formal_Specification_of_Emergent_Behaviours_in_Swarm_Robotic_Systems}, \cite{Symmetry_Reduction_Enables_Model_Checking_of_More_Complex_Emergent_Behaviours_of_Swarm_Navigation_Algorithms}, \cite{A_Matrix_Based_Approach_For_Modeling_Robotic_Swarm_Behavior}, \cite{Verification_of_visibility-based_properties_on_multiple_moving_robots_in_an_environment_with_obstacles}. We have used an integrated tool for modeling and verification of real-time systems, UPPAAL \cite{UPPAAL_in_a_Nutshell}. We have transformed the psuedocode defining the algorithm into a timed automaton. We have explained the implementation of timed automaton in UPPAAL and simplifications of the algorithm. We have described connections between the design choices and expected physical behaviour of the model. We have shown how a composition of models became a system implementing the Alpha algorithm in robot swarm. We have presented how the system is parameterised and mechanisms governing communication. We have introduced the limitations of the system that allowed for proceeding with verification by significantly reducing the state-space size. Finally we have defined and successfully verified properties of the system that increase our confidence int the correctness of implementation. It is important to notice that verified properties do not guarantee the correctness of the Alpha algorithm itself as they focus on the correctness of implementation. Future work may include comparing the impact of modes of concurrency on the property verification and verifying new properties.

