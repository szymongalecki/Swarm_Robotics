\section{Background}

\subsection{UPPAAL}
"UPPAAL is an integrated tool environment for modeling, simulation and verification of real-time systems. It is appropriate for systems that can be modelled as a collection of non-deterministic processes with finite control structure and real-valued clocks, communicating through channels or shared variables. Typical application areas include real-time controllers and communication protocols in particular, those where timing aspects are critical." \cite{UPPAAL_in_a_Nutshell}

Swarm of robots fits within description of a system that can be modelled using UPPAAL. It is a collection of non-deterministic processes - single robots, that are defined using finite control structure, an algorithm that is transformed into a finite state machine. Swarm has to communicate and it can be achieved using channels or shared variables. Finally the emergent behaviour is a result of individual robot behaviour which is influenced by the passing time.


\subsection{Timed automata in UPPAAL}
Timed automata in UPPAAL is based on timed automata defined by Rajeev Alur and David Dill in \cite{Automata_For_Modeling_Real-Time_Systems}, defined in Figure \ref{fig:definition}. UPPAAL builds on top of that definition extending the states of the automata with invariants. Invariant is a condition on the clock which controls how long an automaton can remain in the given state. Edge between states can be decorated with guard, synchronisation action, clock resets and variable assignments. Guard is a condition on the clock or integer variables that needs to be satisfied in order for the transition to be enabled. Synchronisation action is sending a signal to corresponding edge(s) or waiting to receive such a signal in order to synchronise transitions. Clock reset and variable assignment are used to update the state which will be used for edge conditions and determine further transitions. \cite{UPPAAL_in_a_Nutshell}

\begin{figure}[H]
\caption{Definition of timed automaton from \cite{Automata_For_Modeling_Real-Time_Systems}}
A timed automaton is a tuple $(\Sigma, S, S_0, C, E)$ where:\\
$\Sigma$ - input alphabet;\\
$S$ - finite set of automaton states;\\
$S_0 \subseteq S$ - set of start states; \\
$C$ - finite set of clocks; \\
$E \subseteq S \times S [\Sigma \cup {\epsilon}] \times 2^C \times \Phi(C)$ - set of transitions\\\\
An edge in timed automaton is a tuple $\langle s, s', \sigma, \lambda \delta \rangle$, where:\\
$s$ - origin state;\\
$s`$ - destination state;\\
$\sigma$ - input symbol for the transition;\\
$\lambda$ - set of clocks to be reset with this transition;\\
$\delta$ - condition enabling the transition;\\
\label{fig:definition}
\end{figure}
