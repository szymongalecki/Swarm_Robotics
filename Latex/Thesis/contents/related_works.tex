\section{Related Works}
The study in \cite{Dixon2011} verified the correctness of the Alpha algorithm by using model checking on a state-space-reduced system with various modes of concurrency. To address the state explosion problem, the system was limited to 2–3 robots operating on a grid sized 5x5 to 8x8. The concurrency modes examined included synchrony, strict turn-taking, non-strict turn-taking, and fair asynchrony. Synchrony was determined to be the most accurate mode of concurrency for modeling real-world execution. The property "no specific robot will remain disconnected forever" was defined using propositional linear-time temporal logic and verified with the symbolic model checker NuSMV. This property was successfully verified for a system with two robots executing under synchronous concurrency. However, it was falsified for all systems consisting of three robots. The study also suggested the Beta algorithm as a potential topic for future research and highlighted the importance of examining algorithms under different concurrency modes, as they have a significant impact on the results of verification.
\\\\
In \cite{Winfield2005}, the Alpha algorithm was simplified in a manner similar to this work. The paper described the process of working with swarm algorithms within a verification framework. It focused on the Alpha algorithm, which is the direct predecessor of the Beta algorithm. Temporal logic was used to formally specify the emergent behaviors of a robotic swarm system, and the design choices made to the algorithm ensured its feasibility for verification. Two properties were defined but not verified. The first property states, "It is repeatedly the case that for each robot, we can find another robot so that they are connected." The second property states, "Eventually, it will always be the case that every robot is connected to at least $k$ robots", where $k$ is a predefined constant. Notably, $k$ is also a variable used in the Beta algorithm, which extends this approach by incorporating information about the connections of neighboring robots. These properties, with modifications, are also applicable to the Beta algorithm.
\\\\
The paper \cite{Kouvaros2015} presented concepts and notation to automatically determine whether a swarm would exhibit emergent behavior regardless of the number of agents involved. This approach was demonstrated using the Beta algorithm. Although this work is closely related to the algorithm I model, implement, and verify, I was unable to utilize its findings as the concepts were too advanced for my current level of expertise in verification.