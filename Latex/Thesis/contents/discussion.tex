\section{Discussion}
Verification of the Beta algorithm for the asynchronous implementation showed that the algorithm coherence is not guaranteed for the Beta parameter equal to two. That value was used in the original paper \cite{Nembrini2002} however, other values of the Beta parameter were used but their influence was not examined. For swarms of size three, it was shown that all of the robots can become disconnected for at least two steps. For swarms of size four, it was shown that a single robot can become disconnected for at least two steps. The diagnostic trace produced upon verification of property defined in figure \ref{fig:algorithm_asynchronous_properties_2}, was used as a starting point for manual traversal through possible system states using UPPAAL's symbolic simulator. This led to the exploration of positive and negative scenarios for the swarm presented in Figures \ref{fig:lost_connection} and \ref{fig:reconnection}. Due to hardware limitations, it was not possible to verify whether a swarm consisting of four robots could become fully disconnected for at least two steps.
\\\\
While the swarm will react to the formation of bridges described in the original paper \cite{Nembrini2002}, it will not prevent them from forming. This can be observed in the scenario presented in Figure \ref{fig:lost_connection}. This can be partially attributed to the concurrency mode used by the system. Property that describes the situation of a single robot becoming disconnected from the swarm for at least two steps was verified as true for the system operating in asynchronous mode and falsified for the synchronized system. By examining scenarios presented in Figures \ref{fig:lost_connection} and \ref{fig:reconnection} one can speculate that robots that operate in similar frequency are more likely to reconnect. If robots operated in a more uniform way it could be easier for them to reconnect. In the situation of asynchrony, a group of robots may move away before the lost robot realizes that it is disconnected. 
\\\\
When verifying properties of the Beta algorithm for the synchronized system it was found that all of the robots can become disconnected for a single step. However, it was also found that a single robot can't become disconnected for two or more steps. Properties that were verified and falsified can be found in Figures \ref{fig:algorithm_synchronised_properties_true} and \ref{fig:algorithm_synchronised_properties_false}. It can be argued that one of the reasons for this outcome is the reactivity of the system in which all robots move forward at the same time. This could be treated as an indication that the uniformity of robot operations might be a crucial aspect of the algorithm. 
\\\\
There were several limitations affecting the verification process and results, the most significant being the state-space explosion problem. A swarm of size four produced too many states for full verification on a single laptop. To make verification possible, the grid was bounded, and the robot algorithm modified: robots performed a 180-degree turn upon reaching the grid's boundary. This altered the original algorithm and affected its effectiveness, as robots could reconnect by bouncing off boundaries. As a result, defining a property to detect a disconnected robot was straightforward unlike defining one for a connected swarm that did not become connected as a result of interacting with the boundary.