\section{Conclusions}
In this paper, we have presented the process of modeling, implementing, and verifying the Beta algorithm for a robot swarm using timed automata and UPPAAL. Through the implementation and verification of both asynchronous and synchronized implementations, we highlighted the influence of the concurrency mode on the algorithm's effectiveness in maintaining swarm coherence. While the Beta algorithm demonstrated responsiveness to lost connections, challenges emerged in guaranteeing reconnection in the asynchronous system. Furthermore, the reconnection of the swarm cannot be attributed solely to the Beta algorithm, as it is also influenced by the limited area of the grid. These insights highlight the difficulties in verifying swarm robotic algorithms and the influence of robot uniformity on the effectiveness of the algorithm.
\\\\
The results of our verification showed that the synchronization mode significantly affects the coherence of the swarm. Synchronized implementations completely eliminated prolonged disconnections. While this demonstrated the importance of uniformity in robot behavior for maintaining connectivity, a fully synchronized solution defeats the most basic assumption of swarm robotics: decentralization. In contrast, asynchronous systems, while aligning more closely with real-world scenarios, faced challenges with timely reconnection. Additionally, the bounded grid environment introduced limitations that, while necessary for computational feasibility, influenced the outcomes of the algorithm. This highlights the need to balance practical constraints with algorithmic integrity when implementing and verifying swarm robotics algorithms.
\\\\
Future work should focus on developing additional properties to verify the Beta algorithm, ensuring a more comprehensive understanding of its behavior under various conditions. Exploring other approaches to limiting the grid, such as implementing different boundary behaviors or using varying grid sizes, could provide more flexibility in verification while maintaining computational feasibility. Additionally, verification should be expanded to larger swarms with different parameter values, such as varying the Beta parameter and robot count, to assess the algorithm’s scalability. Furthermore, a new mode could be explored where robots are not fully synchronized but operate at the same frequency, balancing the benefits of both presented implementations. This would provide valuable insights into the algorithm's effectiveness.
