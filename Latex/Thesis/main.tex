\documentclass{article}
\usepackage{graphicx}   % Inserting images
\usepackage{listings}   % Code snippets
\usepackage{cite}       % Citing
\usepackage{float}      % Placing figures
\usepackage{amsthm}     % Definitions 
\theoremstyle{definition}                    
\newtheorem{definition}{Definition}[section] 
\usepackage{url}        % Command \url in @misc reference 
\usepackage{lipsum}     % Lorem ipsum
\usepackage{xcolor}     % Text color
\usepackage{gensymb}    % Degree symbol

% Style for pseudo-code
\lstdefinestyle{code}{
  basicstyle=\ttfamily\small,         % Code font size and style
  tabsize=4,                          % Tab space
  frame=single,                       % Frame around the code
  breaklines=true,                    % Automatic line breaking
  postbreak=\mbox{\textcolor{red}{$\hookrightarrow$}\space},    
}

% Style for C++
% Define a custom style for code
\lstdefinestyle{C++}{
    language=C++,                    % Set the language to C++
    basicstyle=\ttfamily\small,      % Code font size and style
    keywordstyle=\color{blue},       % Keywords in blue
    commentstyle=\color{green!50!black}, % Comments in green
    stringstyle=\color{red},         % Strings in red
    tabsize=4,                       % Tab space
    frame=single,                    % Frame around the code
    breaklines=true,                 % Automatic line breaking
    postbreak=\mbox{\textcolor{red}{$\hookrightarrow$}\space},
}


% title, author, date
\title{Implementation and Verification of the Beta Algorithm for Robot Swarm Using Timed Automata and UPPAAL}
\author{Szymon Gałecki - sgal@itu.dk\\KISPECI1SE}
\date{August 2024}


\begin{document}

% title, abstract, table of contents
\maketitle
\begin{abstract}
\noindent
Swarm robotics is a field where multiple simple robots work together to perform tasks without relying on a central controller. This paper focuses on the Beta algorithm, which is designed to keep robots connected through communication with nearby robots, without knowing their exact positions. The Beta algorithm is modeled using timed automata and implemented and verified using UPPAAL. Verification focuses on examining how efficient the Beta algorithm is in preventing the robots of the swarm from disconnecting. Both asynchronous and synchronized versions of the algorithm are explored to examine the influence of the mode of concurrency on the effectiveness of the algorithm.
\end{abstract}
\tableofcontents


% contents
\section{Introduction}
In swarm robotics, multiple robots work together to solve problems by interacting with each other and the environment in a similar way as bees, ants or birds. \cite{Swarm_Robotic_Behaviors_and_Current_Applications}. To enable the interaction within the swarm, first we have to achieve aggregation. There are many algorithms that focus on aggregation task for robot swarm but we have chosen the Alpha algorithm as it was mentioned in reviewed papers \cite{Towards_Temporal_Verification_of_Emergent_Behaviours_in_Swarm_Robotic_Systems}, \cite{On_Formal_Specification_of_Emergent_Behaviours_in_Swarm_Robotic_Systems}, \cite{Symmetry_Reduction_Enables_Model_Checking_of_More_Complex_Emergent_Behaviours_of_Swarm_Navigation_Algorithms}, \cite{A_Matrix_Based_Approach_For_Modeling_Robotic_Swarm_Behavior}, \cite{Verification_of_visibility-based_properties_on_multiple_moving_robots_in_an_environment_with_obstacles} and because it achieves aggregation using a single variable.

Instead of relying on environment and localisation information, it uses physical properties of the signal used for communication. Robot behaviour is solely determined by the change in the number of robots that are in the range of its signal.

The idea behind the Alpha algorithm is that aggregation can be achieved using just the information on the number of robot connections and completely disregarding the environment in which robots exist. The physical reality of such solution would employ a communication technology of a limited range to establish connections to other robots. A single robot behaviour would be determined solely on the number of connections.

In reviewed papers the modeling and implementation parts were either limited or missing. That means that any future work or repeating experiments based on those papers is not possible. This paper will focus on the modeling and implementation aspect of the Alpha algorithm. We will show how an algorithm defined by a pseudocode is transformed into a timed automaton. Using timed automaton will allow for expressing time-dependent behaviours and verifying the model that better captures the real world operation. We will use an integrated tool for modeling and verification, UPPAAL, to implement a timed automaton. We will create system of timed automata that implements the Alpha algorithm for robot swarm. Finally, we will examine the correctness of our implementation by defining and verifying properties of the system.
\section{Background}
This section introduces UPPAAL, a tool for modeling, simulating, and verifying real-time systems. We start with a definition of timed automata, followed by an explanation of its role in modeling real-time behavior. We then outline how UPPAAL implements timed automata. To illustrate these concepts, we use a solution to the mutual exclusion problem as an example. We also explain how to verify properties in UPPAAL using logical quantifiers to define system properties. The example is used to demonstrate the process of modeling and verification in UPPAAL.

\subsection{UPPAAL}
UPPAAL \cite{Larsen1997} is a complete tool for modeling, simulation, and verification of real-time systems. Systems can be modeled as networks of automata and timed automata. A system is composed of one or more models that consist of locations and transitions between locations. Simulation involves traversing the state space to obtain possible paths within the defined system. Simulation is used to interactively check if the system behaves as expected. Verification is realized through model-checking. In the process of verification, properties defined for the system are determined to be valid or not. If the property is found to be false, UPPAAL will produce a diagnostic trace, a path through the system that contradicts the checked property. 

UPPAAL is an appropriate tool to model a robot swarm. A robot swarm is composed of multiple uniform robots. A single robot can be modeled as a timed automaton implementing an algorithm of our choice. A system consisting of multiple uniform timed automata can be used to simulate an algorithm implementation for a robot swarm. This allows us to simulate the robot swarm and perform verification. Verification through model checking can be utilized to verify the correctness of the implementation of the algorithm as well as for the emergence of the desired behavior of the swarm.


\subsection{Timed automata in UPPAAL}
The timed automaton defined in the work of Rajeev Alur and David Dill \cite{Alur1990} is a basis for timed automata used in UPPAAL. Additionally, UPPAAL extends the Definition \ref{def:automaton} of the automaton with invariants and variables of boolean and integer type. The invariant is a progression condition on the system. It states that the system is allowed to stay in a given location only for a specified time before being forced to transition. A transition between locations can be decorated with a guard, a logical condition on the system variables, or clocks. If the logical value of the guard is true, the transition is enabled and disabled otherwise. Transitions can be associated with the synchronization action. Synchronization in UPPAAL is based on handshaking; therefore, one transition is responsible for sending the synchronization signal, and one or more transitions will wait for it. A transition that is waiting for the synchronization signal will remain disabled until the signal is received. This mechanism allows for multiple processes to synchronize their transitions. During transition, it is possible to reset clocks and assign values to variables. These clock and variable values are then used to determine the logical value of the transition guards.

\begin{definition}[Definition of timed automaton \cite{Alur1990}]
A timed automaton is a tuple $(\Sigma, S, S_0, C, E)$ where:\\
$\Sigma$ - input alphabet;\\
$S$ - finite set of automaton states;\\
$S_0 \subseteq S$ - set of start states; \\
$C$ - finite set of clocks; \\
$E \subseteq S \times S [\Sigma \cup {\epsilon}] \times 2^C \times \Phi(C)$ - set of transitions\\\\
An edge in timed automaton is a tuple $\langle s, s', \sigma, \lambda \delta \rangle$, where:\\
$s$ - origin state;\\
$s`$ - destination state;\\
$\sigma$ - input symbol for the transition;\\
$\lambda$ - set of clocks to be reset with this transition;\\
$\delta$ - condition enabling the transition;\\
\label{def:automaton}
\end{definition}


\subsection{Modeling in UPPAAL}
To explain the process of modeling in UPPAAL we will use one of the example models described in the UPPAAL tutorial \cite{SmallTutorial2009}. The example model implements Gary L. Peterson's solution to the mutual exclusion problem \cite{Peterson1981}. Figure \ref{fig:mutex_code} presents his solution to the problem of two processes sharing access to the critical section. A critical section is a part of code that must be executed only by a single process at a time \cite{Raynal2012}.

% Pseudo-code for mutex
\begin{figure}[H]
\caption{Peterson’s mutual exclusion algorithm 
\label{fig:mutex_code}
\cite{SmallTutorial2009}, \cite{Peterson1981}}
\begin{tabular}{|p{0.5\textwidth}|p{0.5\textwidth}|}
\hline
\textbf{Process 1} & \textbf{Process 2} \\
\hline
\begin{lstlisting}[basicstyle=\ttfamily]
req1=1;
turn=2;
while(turn!=1 && req2!=0);
// critical section:
job1();
req1=0;
\end{lstlisting}
&
\begin{lstlisting}[basicstyle=\ttfamily]
req2=1;
turn=1;
while(turn!=2 && req1!=0);
// critical section:
job2();
req2=0;
\end{lstlisting}
\\
\hline
\end{tabular}
\end{figure}

\noindent
Peterson's solution of the mutual exclusion algorithm consists of two symmetrical processes. Each process requests access to the critical section and then sets a flag indicating the other process's turn to access. A process will continuously wait to access the critical section until its turn or until the other process no longer requests access. After accessing the critical section and completing the associated work, the process will indicate that it no longer requests access. This solution guarantees fairness as no process will be indefinitely denied access to the critical section.


% Mutex implementation in UPPAAL
\begin{figure}[H]
\caption{Mutex automata in UPPAAL \cite{SmallTutorial2009}}
\includegraphics[width=\textwidth]{images/mutex.png}
\label{fig:mutex_uppaal}
\end{figure}

\noindent
To model Peterson's solution of the mutual exclusion algorithm we will represent two processes as separate automata. The automaton on the left in Figure \ref{fig:mutex_uppaal} will represent \texttt{Process 1} from Figure \ref{fig:mutex_code} and the right automaton will represent \texttt{Process 2}. Both automata have the same set of locations, namely, \texttt{idle}, \texttt{want}, \texttt{wait}, and \texttt{CS}. Location \texttt{idle} represents the state of the process in which it does not request access to the critical section. Location \texttt{want} represents the state of the process after requesting access to the critical section. Location \texttt{wait} indicates that the process set the turn to access the critical section to the other process. Automaton will remain in location \texttt{wait} until one of the guard conditions gets satisfied and enables the transition to location \texttt{CS}. Location \texttt{CS} is a critical section of the process. Upon transition from location \texttt{CS} to \texttt{idle} a process will no longer request access to the critical section. 


\subsection{Verifying properties in UPPAAL}
A subset of timed computation tree logic is used to express properties of the system that are verified by UPPAAL's model checking engine \cite{Bengtsson2004}. The system of timed automata is unfolded into a tree with states and transitions. Paths of such a tree are traversed by the model checking engine to determine whether defined system properties are true. System properties must be specified using logical quantifiers presented in Definition \ref{def:quantifiers}. Letters \texttt{A} and \texttt{E} are used to quantify over paths while symbols \texttt{[]} and \texttt{<>} are used to quantify over states within a path. Letter \texttt{A} is used to express property that has to hold for all paths and letter \texttt{E} is used for property that holds for at least one path. Analogically, symbol \texttt{[]} is used to express that all states within a path must satisfy the property, and symbol \texttt{<>} is used to express that there is at least one state within the path that satisfies the property.

% Logical quantifiers in UPPAAL
\begin{definition}[Logical quantifiers in UPPAAL \cite{Bengtsson2004}]
The formulas should be one of the following forms\\
- \texttt{A[]$\phi$} -- Invariantly $\phi$.\\
- \texttt{E<> $\phi$} -- Possibly $\phi$.\\
- \texttt{A<> $\phi$} -- Always Eventually $\phi$.\\
- \texttt{E[] $\phi$} -- Potentially Always $\phi$.\\
- \texttt{$\phi$ --> $\psi$} -- $\phi$ always leads to $\psi$.\\
where $\phi, \psi$ are local properties that can be checked locally on a state, i.e. boolean expressions over predicates on locations and integer variables, and clock constraints.
\label{def:quantifiers}
\end{definition}

\noindent
To present the verification process in UPPAAL we specified and successfully verified properties of Mutex implementation in Figure \ref{fig:mutex_verification}. The first property states that for all paths and all states of those paths, there is never a situation where both automata are accessing the critical section at the same time. This is a safety property that verifies whether mutual exclusion, the main objective of the algorithm, is achieved. The second and third properties are liveness properties. They state that there exists a path with a state which enables a process to access a critical section. Successful verification of all three properties means that the solution guarantees mutual exclusion and fairness. Fairness in this context means that there exists a path through the system which results in a process accessing the critical section.

% Successfully verified properties for mutex
\begin{figure}[H]
\caption{Successfully verified properties for mutex \cite{SmallTutorial2009}}
\label{fig:mutex_verification}
\begin{lstlisting}[style=code]
A[] not (P1.CS and P2.CS)
E<> P1.CS
E<> P2.CS
\end{lstlisting}    
\end{figure}
\section{Methods}



\subsection{Aggregation}
We will use Manuele Brambilla's \cite{Brambilla2013} taxonomy of collective swarm behaviors, in Figure \ref{fig:taxonomy}, to show the context for the modeled aggregation behavior. Behaviors categorized as spatially-organizing distribute robots and objects in space. These behaviors serve as fundamental building blocks for more advanced swarm behaviors, as they enable robots to connect, communicate, and interact with one another. Aggregation, the simplest of collective behaviors, groups all robots of a swarm in a region of the environment. If we assume that the robots' environment is unbounded we will have two design choices. The aggregation algorithm will either have to allow a robot to reconnect with the group or never let it disconnect from it. The first approach involves access to external information about the position of the swarm e.g. robot coordinates. The second approach relies on robots being initially connected.

% Taxonomy
\begin{figure}[H]
\caption{Taxonomy of collective swarm behaviours \cite{Brambilla2013}}
\includegraphics[width=\textwidth]{images/taxonomy.png}
\label{fig:taxonomy}
\end{figure}


\newpage
\subsection{Beta Algorithm}
The Beta algorithm is a robot swarm aggregation algorithm introduced by Julien Nembrini. It is his next iteration of the aggregation algorithm that relies on situated communication. Robots do not have information on their environment and the exact location of other robots. They are connected if they are within communication distance of the physical signal they are using. In the previous approach of the Alpha algorithm, robot movement was determined by the number of connections to other robots. In the Beta algorithm robot uses information about its connections as well as about connections of its connections. Robot when losing a connection will consider how many of its neighbors, are connected to the disconnected robot. If the number of neighbors that are connected to that robot is equal to or lower than the beta parameter, the robot will turn back in an attempt to reconnect. This aims at preventing a previously observed situation where a single robot would disconnect from the swarm without triggering its reaction as the remaining robots would have a satisfactory number of connections. In the event of high swarm congestion, a robot will choose a random direction. 
TBC

% Pseudo-code for Beta algorithm
\begin{figure}[H]
\caption{Pseudo-code for Beta algorithm \cite{Nembrini2002}}
\begin{lstlisting}[style=code]
Create list of neighbours for robot, Nlist
k = number of neighbours in Nlist
i = 0

loop forever {
	i = i modulo cadence

	if (i = 0) {
		Send ID message

		Save copy of k in LastK
		Set reaction indicator Back to FALSE
		k = number of neighbours in Nlist
		Create LostList comparing Nlist and OldList

		for (each robot in LostList) {
			Find nShared, number of shared neighbours
			if (nShared <= beta) {
				Set reaction indicator Back to TRUE
			}
		}

		if (Back = TRUE) {
			turn robot through 180 degrees
		}
		else if (k > LastK) {
			make random turn
		}
		
		Save copy of Nlist in Oldlist
	}
	Steer the robot according to state
	Listen for calls from robots in range
	Grow Nlist with neighbours IDs and connection info

	i++
}
\end{lstlisting}
\label{fig:pseudocode}
\end{figure}


\subsection{Movement}



\subsection{Connection}



\subsection{System variables}



\subsection{Automaton}

% Automaton
\begin{figure}[H]
\caption{Timed automaton for the Beta algorithm}
\includegraphics[width=\textwidth]{images/beta.png}
\label{fig:automaton}
\end{figure}

\section{Implementation}
In this section, we will describe the implementation of the timed automaton model from the previous section, shown in Figure \ref{fig:automaton}. Our tool of choice is UPPAAL, an integrated solution for modeling, simulation, and verification of timed automata. We will present the implementation of limitations imposed on the model in the following subsection.  Model variables representing its state will be described along with the most important functions.


\subsection{Model}
Implementation of the timed automaton, shown in Figure \ref{fig:implementation}, has an additional state called \texttt{grid} and global functions that abstract away the underneath complexity. This state is a consequence of imposing limitations on the environment size defined in the previous section. This allows us to reduce the state space size and perform verification. It is the biggest change made to the original algorithm, defined in Figure \ref{fig:pseudocode}. In that state, it is determined whether a robot has reached the boundary of the grid. If yes, the robot will transition to the state \texttt{turn\_180} and change its direction by 180 degrees as it cannot continue moving forward outside of its environment. If a robot has not reached the boundary of its environment it will transition to the \texttt{if} state where it will follow the original rules of the algorithm.

\begin{figure}[H]
\caption{Asynchronous implementation of the Beta algorithm}
\includegraphics[width=\textwidth]{images/implementation.png}
\label{fig:implementation}
\end{figure}

\noindent
The implementation of the Beta algorithm presented in Figure \ref{fig:implementation} is asynchronous as stated by the author of the algorithm. Robots can move at different pace and time, independent of each other. We also implemented a synchronized version of the Beta algorithm to investigate the influence of the concurrency mode on verification results. The synchronized implementation is presented in Figure \ref{fig:implementation_synchronised}, and its synchronization mechanism in Figure \ref{fig:implementation_synchronised_barrier}. The synchronized version enforces that all robots move at the same time. The synchronization mechanism, which we will call barrier, has a variable \texttt{n} initialized to the number of robots in the swarm. Each time a robot transitions to the \texttt{forward} state it will use a synchronization channel \texttt{done} to notify the barrier. The barrier will decrement its variable \texttt{n} upon receiving a signal from a robot. When the value of \texttt{n} reaches 0, all robots will be blocked in the \texttt{forward} state. The barrier then will send a signal using synchronization channel \texttt{step} and reset the value of the variable \texttt{n} to the number of robots in the swarm. Robots waiting in the \texttt{forward} state will receive this signal from the \texttt{step} channel and transition to \texttt{grid} state at the same time. They will then progress to the \texttt{forward} state in the random order, possibly changing its direction. The order in which they reach the \texttt{forward} state does not influence swarm behavior. A robot decides whether to change direction based on the state of the swarm. The state of the swarm, most importantly robot positions, will not change until the next collective step forward. In other words, the direction of the robot is not a variable that influences the behavior of another robot, unlike its position. 

\begin{figure}[H]
\caption{Synchronised implementation of the Beta algorithm}
\includegraphics[width=\textwidth]{images/implementation_synchronised.png}
\label{fig:implementation_synchronised}
\end{figure}

\begin{figure}[H]
\caption{Synchronisation mechanism}
\centering
\includegraphics[width=0.5\textwidth]{images/implementation_synchronised_barrier.png}
\label{fig:implementation_synchronised_barrier}
\end{figure}

\subsection{Movement}



\subsection{Connection}



\subsection{System variables}

\section{Results}
\subsection{Verifying implementation}


\subsection{Verifying the algorithm}

\section{Related works}

Paper  \cite{Towards_Temporal_Verification_of_Emergent_Behaviours_in_Swarm_Robotic_Systems} tried to verify the correctness of the Alpha algorithm by model checking the state-space reduced system with different modes of concurrency. System was reduced to 2-3 robots and grid of size 5x5-8x8 to avoid the state explosion problem. Examined modes of concurrency were synchrony, strict turn taking, non-strict turn taking and fair asynchrony. Synchrony was found to be the most accurate mode of concurrency for modeling real execution. Property 'no specific robot will remain disconnected forever' was defined using propositional linear-time temporal logic and verified using symbolic model checker -NuSMV. The property was succesfully verified for a system consisting of two robots executing with synchronous mode of concurrency. The property was falsified for all systems consisting of three robots.

In \cite{On_Formal_Specification_of_Emergent_Behaviours_in_Swarm_Robotic_Systems} the Alpha algorithm was simplified in a similar way as in this works. It used temporal logic to formally specify emergent behaviours of a robotic swarm system. Two such properties were defined but not verified. Property 1 - "It is repeatedly the case that for each robot, we can find another robot so that they are connected". Property 2 - "Eventually it will always be the case that every robot is connected to at least $k$ robots, where $k$ is a pre-defined constant".

Work \cite{Formal_Verification_of_Probabilistic_Swarm_Behaviours} modeled the foraging scenario for the robot swarm using discrete-time Markov chain. It analysed global swarm behaviour by defining properties in probabilistic computation tree logic and verifying them with probabilistic model checker - PRISM. The focus of the work was to examine the parameters of the system that would lead to energy conservation by the robot swarm. Tasks within the foraging scenario required consuming energy, but the finding a food item was assumed to provide energy. Since PRSIM is a probabilistic model checker, the defined property, "For an arbitrary number of robots and food finding probability the swarm energy is equal to or greater than the initial energy from a time point $t_A$”, yielded a valid probability. Results were computed for a range of system parameters to model the energy levels within a swarm over time.  

\section{Conclusions}
In this paper we have presented the detailed implementation of the Alpha algorithm \cite{Minimalist_Coherent_Swarming_of_Wireless_Networked_Autonomous_Mobile_Robots} using network of timed automata. The Alpha algorithm has been chosen because of its presence in reviewed literature \cite{On_Formal_Specification_of_Emergent_Behaviours_in_Swarm_Robotic_Systems}, \cite{Towards_Temporal_Verification_of_Emergent_Behaviours_in_Swarm_Robotic_Systems}. We have used an integrated tool for modeling and verification of real-time systems, UPPAAL \cite{UPPAAL_in_a_Nutshell}. We have transformed the psuedocode defining the algorithm into a timed automaton. We have explained the implementation of timed automaton in UPPAAL and simplifications of the algorithm. We have described connections between the design choices and expected physical behaviour of the model. We have shown how a composition of models became a system implementing the Alpha algorithm in robot swarm. We have presented how the system is parameterised and mechanisms governing communication. We have introduced the limitations of the system that allowed for proceeding with verification by significantly reducing the state-space size. Finally we have defined and successfully verified properties of the system that increase our confidence int the correctness of implementation. It is important to notice that verified properties do not guarantee the correctness of the Alpha algorithm itself as they focus on the correctness of implementation. Future work may include comparing the impact of modes of concurrency on the property verification and verifying new properties.






% bibliography
\bibliography{refs}
\bibliographystyle{plain}


\end{document}
